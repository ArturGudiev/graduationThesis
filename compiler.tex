\documentclass[14pt]{extarticle}
\usepackage[
left=25mm,
top=20mm,
right=15mm,
bottom=20mm,
]{geometry}

%\usepackage{graphicx}
\usepackage[pdftex]{graphicx}
\usepackage[utf8x]{inputenc}
\usepackage[russian]{babel}
\usepackage[T1]{fontenc}
\usepackage{float}
\usepackage{listings}
\usepackage{cite}
\usepackage{hyperref}
\usepackage{etoolbox}
\usepackage{indentfirst}
\usepackage[linesnumbered,boxed]{algorithm2e}
%\sloppy

\lstset{
	sensitive=true,
	basicstyle=\small,
	keywordstyle=\color{black},
	commentstyle=\scriptsize\rmfamily,
	keywordstyle=\ttfamily\underbar,
	identifierstyle=\ttfamily,
	basewidth={0.5em,0.5em},
	columns=fixed,
	fontadjust=true,
	literate={->}{{$\to$}}1
}

\makeatletter
%\renewcommand{\@biblabel}[1]{#1.} % Заменяем библиографию с квадратных скобок на точку:
\makeatother
\gappto\captionsrussian{\renewcommand{\contentsname}{Оглавление}}
\renewcommand\baselinestretch{1.5}
\renewcommand{\lstlistingname}{Листинг}

\begin{document}
	
	\begin{titlepage}
		\thispagestyle{empty}
		\def\baselinestretch{1.0}
		\begin{center}
			{САНКТ-ПЕТЕРБУРГСКИЙ ГОСУДАРСТВЕННЫЙ УНИВЕРСИТЕТ \\ \vskip 0.3em {\large Математико-механический факультет \\ \vskip 0.7em{\large Кафедра системного программирования \\}}}
			\vspace*{0.15\textheight}
			{\large Гудиев Артур Владимирович}
			
			\vskip 2em
			{\LARGE Реализация примитвов и оконного менеджера для построения пользователских интерфейсов на языке PostScript}
			
			\vskip 1em
			{\large Дипломная работа} \\
			\vskip 2em
			{\normalsize \raggedleft 
				Допущена к защите.\\
				Зав. кафедрой:\\
				д.ф.-м.н., проф. А.Н. Терехов
				\\[2em]
				Научный руководитель:\\
				к.ф.-м.н. Д.Ю. Булычев
				\\[2em]
				Рецензент:\\
				Д.В. Кознов
				\\[2em]
				%Неизвестно \\
				\vspace*{0.08\textheight}
				{\centering Санкт-Петербург \\ 2015}
			}
		\end{center}
	\end{titlepage}
	\begin{titlepage}
		\thispagestyle{empty}
		\def\baselinestretch{1.0}
		\begin{center}
			{SAINT-PETERSBURG STATE UNIVERSITY \\ \vskip 0.3em {\large Mathematics \& Mechanics Faculty \\ \vskip 0.7em{\large Department of Software Engineering \\}}}
			\vspace*{0.15\textheight}
			{\large Artur Gudiev}
			
			\vskip 2em
			{\LARGE Window manager and GUI primitives for user interface implementation in PostScript}
			
			\vskip 1em
			{\large Graduation Thesis} \\
			\vskip 2em
			{\normalsize \raggedleft 
				Adnitted for defence.\\
				Head of the chair:\\
				professor  Andrey Terekhov
				\\[3em]
				Scientific supervisor:\\
				Dmitri Boulytchev
				\\[3em]
				Reviewer:\\
				Dmitri Koznov
				\\[2em]
				%Неизвестно \\
				\vspace*{0.08\textheight}
				{\centering Saint-Petersburg \\ 2015}
			}
		\end{center}
	\end{titlepage}
	
	\tableofcontents
	\thispagestyle{empty} 
	\pagebreak
	
	
	
	\section*{Введение}
	\addcontentsline{toc}{section}{Введение}
	Язык PostScript - это графический интерпретируемый язык программирования, создавашийся с целью представления графики в машинонезависимой форме. Посредством графических операторов языка PostScript можно отобразить на экране прямые и кривые линии, залить цветом область, определить область рисования, задать графические параметры.  
	
	Графический интерфейс пользователя - это разновидность пользовательского интерфейса, в котором элементы интерфейса представлены в виде графических примитивов. Визуальные интерфейсы пользователя упрощают работу с программами, делая ее наглядной. Язык PostScript обладает базовыми графическими возможностями для отображения внешнего вида графических интерфейсов.
	
	Оконный менеджер — это приложение, управляющее размещением окон и определяющая их внешний вид в оконной системе графического интерфейса. Оконные менеджеры работают на основе существующей оконной системы. Кроме того оконный менеджер включает в себя и визуальные эффекты, проявляющиеся во время работы с окнами. Обычно оконный менеджер привязан к конкретной операционной системе. Язык PostScript позволяет нарисовать такие объекты, как элементы графического интерфейса пользователя и визуальные эффекты оконного менеджера.
	
	Ранее в рамках проекта лаборатории JetBrains был реализован интерпретатор графического языка PostScript, однако с его помощью было трудно создавать графические интерфейсы и оконный менеджер из-за того, что, например, в PostScript не поддерживается механизм обработки событий. Для упрощения реализации этой возможности планируется расширить язык PostScript. Данная работа ведется по трем направлениям: оптимизация интерпретатора (Д. Поздин), обработка событий (Р. Макулов) и реализация графических примитивов и оконного менеджера (А. Гудиев).
	
	Целью данной дипломной работы является реализация графических примитивов и оконного менеджера для создания пользовательских интерфейсов на языке PostScript.
	
	\pagebreak
	\section{Обзор}
	\subsection{ Описание существующих решений }
		\subsubsection{Qt Quick}
		\subsubsection{Java Swing}
		\subsubsection{ Недостатки существующих решений }
		
	\subsection{Описание используемых инструментов }
		\subsubsection{JVM}
		\subsubsection{Java Swing}
		\subsubsection{ Интерпретатор JB }
		Интерпретатор PostSc
		
	\subsection{ Проект рабочей группы интерпретатора PostScript }
Задача реализации интерпретатора графического языка PostScript решается в рамках проекта компании JetBrains. 
		
		Проект можно разделить на задачи:
		\begin{itemize}
		\item Оптимизация интерпретатора
		\item Реализация механизма обработки событий
		\item Реализация примитивов и оконного менеджера		
		\end{itemize}	
	
	\pagebreak
	\section{Примитивы графической библиотеки}
	\subsection{Кнопка}
	
		\begin{figure}[h]
		\begin{center}
		\begin{minipage}[h]{0.4\linewidth}
		\includegraphics[width=180pt]{pictures/close1.png}
		\caption{ Кнопка} %% подпись к рисунку
		\label{ris:b1} %% метка рисунка для ссылки на него
		\end{minipage}
		\hfill 
		\begin{minipage}[h]{0.4\linewidth}
		\includegraphics[width=180pt]{pictures/close2.png}
		\caption{Нажатая кнопка}
		\label{ris:b2}
		\end{minipage}
		\end{center}
		\end{figure}	
		
	\subsection{Флажок}
	Флажок - примитив, который содержит только два состояния.
		\begin{figure}[h]
		\begin{center}
		\begin{minipage}[h]{0.4\linewidth}
		\includegraphics[width=180pt]{pictures/checkBox1.png}
		\caption{Флажок} %% подпись к рисунку
		\label{ris:b1} %% метка рисунка для ссылки на него
		\end{minipage}
		\hfill 
		\begin{minipage}[h]{0.4\linewidth}
		\includegraphics[width=180pt]{pictures/checkBox2.png}
		\caption{Отмеченный флажок}
		\label{ris:b2}
		\end{minipage}
		\end{center}
		\end{figure}	
			
	Флажок важен для интерфейсов.	
		
	\pagebreak		
	\subsection{Поле со списком}
	Поле со списком предоставляет пользователю варианты.
	
		\begin{figure}[h]
		\begin{center}
		\begin{minipage}[h]{0.4\linewidth}
		\includegraphics[width=180pt]{pictures/comboBox1.png}
		\caption{ Поле со списком} %% подпись к рисунку
		\label{ris:b1} %% метка рисунка для ссылки на него
		\end{minipage}
		\hfill 
		\begin{minipage}[h]{0.4\linewidth}
		\includegraphics[width=180pt]{pictures/comboBox2.png}
		\caption{Раскрытое поле со списком}
		\label{ris:b2}
		\end{minipage}
		\end{center}
		\end{figure}
		
	\subsection{Список}
	Список - примитив с вариантами.
		\begin{figure}[h]
		\center{\includegraphics[width=180pt]{pictures/listBox.png}}
		\caption{Список}
		\label{ris:image}
		\end{figure}	
	
	\subsection{Метка}
		Метка - это текст по сути.
		\begin{figure}[h]
		\center{\includegraphics[width=180pt]{pictures/label.png}}
		\caption{ Метка }
		\label{ris:image}
		\end{figure}	

	\subsection{Поле редактирования}
		Поле редактирования - это место, куда печатается текст.
		\begin{figure}[h]
		\center{\includegraphics[width=180pt]{pictures/textField.png}}
		\caption{ Поле редактирования }
		\label{ris:image}
		\end{figure}	
	\subsection{Радиокнопка}
		Кнопка, два варианта включения.
		\begin{figure}[h]
		\begin{center}
		\begin{minipage}[h]{0.4\linewidth}
		\includegraphics[width=90pt]{pictures/toggleButton1.png}
		\caption{ Включенная радиокнопка} %% подпись к рисунку
		\label{ris:b1} %% метка рисунка для ссылки на него
		\end{minipage}
		\hfill 
		\begin{minipage}[h]{0.4\linewidth}
		\includegraphics[width=90pt]{pictures/toggleButton2.png}
		\caption{Выключенная радиокнопка}
		\label{ris:b2}
		\end{minipage}
		\end{center}
		\end{figure}
	\subsection{Окно}
	Окно - важный элемент интерфейсов пользователя.
		\begin{figure}[h]
		\center{\includegraphics[width=180pt]{pictures/window.png}}
		\caption{ Окно }
		\label{ris:image}
		\end{figure}	
	\pagebreak
	\section{Реализация оконного менеджера}
	\subsection{ Оконная система }
	\subsection{ Архитектура оконного менеджера}
	\pagebreak
	\section{Тестирование и демонстрационный примеры}

	\pagebreak
	
	\section*{Заключение}
	\addcontentsline{toc}{section}{Заключение}
	
	В рамках дипломной работы получены следующие результаты:
	\begin{itemize}
		\item Добавлены примитивы в графическую библиотеку  PostScript.
		\begin{itemize}
			\item Кнопка
			\item Флажок
			\item Поле со списком
			\item Список
			\item Метка
			\item Поле редактирования
			\item Радиокнопка
			\item Окно
		\end{itemize}
		\item Реализован оконный менеджер, интегрированный с графической библиотекой.
		\item Проведено тестирование оконного менеджера на демонстрационных примерах
	\end{itemize}
	

	
	\pagebreak
	\bibliographystyle{ugost2008ls}
	
	
	\begin{thebibliography}{}
		
		\bibitem{PLRM}
		Спецификация языка Postscript. PostScript Language reference. \\
		Adobe Systems. 1999\\
		\url{http://www.adobe.com/products/postscript/pdfs/PLRM.pdf}
		
		\bibitem{jvms}
		Tim Lindholm, Frank Yellin, Gilad Bracha, Alex Buckley.
		The Java Virtual Machine Specification.
		Java SE 7 Edition, 2013. \\
		\url{docs.oracle.com/javase/specs/jvms/se7/jvms7.pdf}
		
		\bibitem{cormen}
		Томас Х. Кормен, Чарльз И. Лейзерсон, Рональд Л. Ривест, Клиффорд Штайн.
		Алгоритмы: построение и анализ.
		Второе издание, 2006
		
	\end{thebibliography}
\end{document}
